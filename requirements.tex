\section{DM requirements on user batch} \label{sec:requirements}

The DM requirements \citeds{LSE-61} which flow down from the System level requirements \citedsp{LPM-17,LSE-29,LSE-30},
has  a few requirements which require some sort of user driven processing.
Many of these pertain to {\em user generated data products}\footnote{Previously known as Level 3 data products.}.
In the era of pre-operations many of these user generated products will be created and may exist at IDACs.
The a broad summary of requirements pertaining to user batch and more broadly to the topic of user generated products and services is covered on confluence\footnote{Should this be in a DMTN or put here ?} \url{https://confluence.lsstcorp.org/display/DM/Level+3+Definition+and+Traceability}{Level 3 Definition and Traceability}.

The requirements most pertinent to user batch from \cite{LSE-61} are :

\begin{itemize}
\item DMS-REQ-0119: DAC resource allocation for Level 3 processing
\item DMS-REQ-0121: Provenance for Level 3 processing at DACs
\item DMS-REQ-0125: Software framework for Level 3 catalog processing
\item DMS-REQ-0128: Software framework for Level 3 image processing
\item DMS-REQ-0123: Access to input catalogs for DAC-based Level 3 processing
\item DMS-REQ-0127: Access to input images for DAC-based Level 3 processing
\item DMS-REQ-0124: Federation with external catalogs
\item DMS-REQ-0106: Coadded Image Provenance
\item DMS-REQ-0335: PSF-Matched Coadds
\end{itemize}


A set of potential use cases for user batch have been proposed in \citeds{DMTN-202}.
This document also provided a summary list which is not dissimilar to the requirements above:

\begin{enumerate}
\item The user computing capability should allow running in bulk over catalog data.
    \item The user computing capability should allow running in bulk over image data.
    \item The system capacity is defined as an “amount of computing capacity equivalent to at least userComputingFraction (10\%) of the total LSST data processing capacity (computing and storage) for the purpose of scientific analysis of LSST data and the production of Level 3 Data Products by external users”.
    \item We have to provide a software framework to facilitate both catalog- and image-based user computation, which has to support systematic runs over collections of data and has to preserve provenance.
    \item The framework(s) has/have to support re-running standard computations from the pipelines in addition to running more free-form user jobs.
    \item There has to be a resource allocation mechanism to allow users to be given quotas, which can be modified per-user. The association of quotas with defined groups of users (e.g., ad-hoc collaborations and/or formal Science Collaborations) would be a useful further capability.

\end{enumerate}
